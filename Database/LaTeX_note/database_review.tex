\documentclass[a4paper, 12pt]{article}
\usepackage{amsmath, amssymb}
\usepackage{geometry}  % 控制页面边距
\usepackage{graphicx}  % 插入图片 
\usepackage{hyperref}  % 目录可以跳转
\geometry{left=2.5cm, right=2.5cm, top=2.5cm, bottom=2.5cm}

\usepackage{fancyhdr}  % 页眉和页脚
\pagestyle{fancy}
\fancyhf{}  % 清除默认设置
\fancyhead[L]{Database Review Notes}  % 左侧页眉内容
\fancyhead[R]{}  % 确保右侧页眉清除
\fancyfoot[C]{\thepage}  % 页脚显示页码

\setlength{\parindent}{0pt} % 取消缩进

% 标题与图片
\title{
    \includegraphics[width=0.8\textwidth]{Yanami_ep1.JPG}\\  % 确保图片路径正确
    Database REVIEW
}
\author{Kinoko}
\date{\today}

\begin{document}

\pagenumbering{gobble}  % 禁用页码

\begin{titlepage}  
    \maketitle
\end{titlepage}

\pagenumbering{roman}  % 使用罗马数字
\fancyhead[R]{}  % 禁用右侧页眉

% 插入目录页
\tableofcontents
\newpage  % 目录单独一页

\pagenumbering{arabic}  % 设置为阿拉伯数字

\section*{Course Syllabus}  % 无编号的章节
\fancyhead[R]{September 23, 2024}  % 手动设置日期
\addcontentsline{toc}{section}{Course Syllabus}  % 手动添加到目录中
    \begin{center}
        \begin{enumerate}
        \renewcommand{\labelenumi}{\Roman{enumi}.}  % 将编号改为大写罗马数字
        \item General Introduction
        \item Relational Model
        \item SQL
        \item Database Security
        \item Database Integrity
        \item Relational Database Theory 
        \item Relational Database Design
        \item Query Processing and Optimization
        \item Database Recovery
        \item Concurrency Control
    \end{enumerate}
    \end{center}
    

\newpage

% 设置章节从 1 开始编号
\setcounter{section}{0}

\section{General Introduction}
\fancyhead[R]{September 23, 2024}  % 手动设置日期
    \subsection{What is data?}
        \begin{itemize}
            \item Data is just the data and information that can be stored.
            \item it is typically \textbf{unprocessed} and \textbf{raw}.
            \item Once we have put our data into \textbf{context}, \textbf{data is transformed into information}, which
            is then used to make decisions.
        \end{itemize}
    
    \subsection{Databases, Data, and Information}
        Database
        \begin{itemize}
            \item Collection of data organized in a manner that allows access, retrieval, and use of
        \end{itemize}

        Data
        \begin{itemize}
            \item Collection of unprocessed items
            \item Text, Numbers, Images, Audio, Video
        \end{itemize}

        Information
        \begin{itemize}
            \item Processed data
            \item organized, meaningful, useful
        \end{itemize}

    \subsection{What is Database?}
    Database = A large collection of related
    data
    \subsection{Database Management System (DBMS)}
    \begin{itemize}
        \item Compilation of related data
        \item A collection of programs for accessing data
        \item A database management system (DBMS) stores information about a specific
        organization.
        \item Reliable and easy to use
    \end{itemize}

    \subsection{DBMS Core Functions}
    DBMS gives us an interface or tool to carry out a variety of tasks, including building databases,
    storing data in them, updating data, creating tables in the databases, and much more.
    \begin{itemize}
        \item Defining a particular database in terms of\\
        \textbf{its data types ,structures , and constraints} 
        \item Manipulating the database
        \begin{itemize}
            \item Retrieval: Querying, generating reports
            \item Modification: Insertions, deletions and updates to its content
            \item Accessing the database through Web applications
        \end{itemize}
        \item Processing and sharing by a set of concurrent users and application programs
        yet, keeping all data valid and consistent.
    \end{itemize}

    \subsection{DBMS Con.}
    Additionally, DBMS offers security and protection to the
    databases.\\
    Maintaining data consistency when multiple users are
    present.\\
    Example of DBMS software: MySql, Oracle, Ms SQL Server

    \subsection{Types of Database}
    \begin{itemize}
        \item Traditional Applications:
        \begin{itemize}
            \item Numeric and Textual Databases
        \end{itemize}
        \item More Recent Applications:
        \begin{itemize}
            \item Multimedia Database (images, audio, video…)
            \item Geographic Information Systems (GIS): Store and analyze maps, weather data, and satellite images
        \end{itemize}
        \item Data Warehouses and online analytical processing (OLAP) systems
        \begin{itemize}
            \item Extract and analyze useful business information from very
            large databases.
            \item Support decision making.
        \end{itemize}
        \item Real-time and Active Databases
        \begin{itemize}
            \item A real-time database is a database system that processes
            data in real time to handle workloads that are constantly
            changing.
        \end{itemize}
    \end{itemize}

    \subsection{The Database approach's main characteristics}
    \begin{itemize}
        \item Data Abstraction
            \begin{itemize}
                \item A data model hides storage details while providing users with a conceptual view of the database.
                \item Programs refer to the data model constructs rather than data storage details.
            \end{itemize}
        \item Multiple data views are supported
            \begin{itemize}
                \item Each user may see a different view of the database that only shows the data that is relevant to them.
            \end{itemize}
        \item Data sharing and transaction processing for multiple users
            \begin{itemize}
                \item Allowing multiple users to access and update the database at the same time.
                \item The recovery subsystem ensures that the effect of each completed transaction is permanently recorded in the database.
                \item Database applications rely heavily on OLTP (Online Transaction Processing). 
                Hundreds of concurrent transactions can be executed per second as a result of this.                
            \end{itemize}
    \end{itemize}

\newpage
\section{Week 3}
\fancyhead[R]{September 30, 2024}  % 手动设置日期
    \subsection{placeholder}

    

\end{document}
